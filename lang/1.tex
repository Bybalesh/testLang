\documentclass{article}
\usepackage{amsmath} % Пакет для работы с математическими формулами
\usepackage{graphicx} % Пакет для вставки изображений

\begin{document}

\title{Пример документа на LaTeX}
\author{Ваше Имя}
\date{\today} % Автоматическая вставка сегодняшней даты

\maketitle

\section{Введение}
Это пример документа, набранного в системе \LaTeX. Она позволяет создавать документы высокого качества, особенно те, которые содержат математические формулы.

\section{Формулы}
Математические формулы можно набирать прямо в тексте. Например, формула квадрата суммы выглядит так:

\[
(a + b)^2 = a^2 + 2ab + b^2
\]

Также можно вставлять изображения:

\begin{figure}[h!]
\centering
\includegraphics[width=0.5\textwidth]{example_image.png}
\caption{Пример изображения.}
\label{fig:example_image}
\end{figure}

\section{Заключение}
Как видите, LaTeX предоставляет мощные инструменты для создания профессиональных документов.

\end{document}

\maketitle

\begin{abstract}
Это краткое содержание вашего документа.
\end{abstract}

\section{Введение}
Здесь вы описываете введение к вашему документу.

\subsection{Подраздел введения}
Более детальная информация о введении.

\section{Основная часть}
Основной текст вашего документа.

\subsection{Первый подраздел основной части}
Описание первого подраздела.

\subsubsection{Мелкий подраздел}
Еще более подробная информация.

\section{Заключение}
Заключение вашего документа.

\end{document}


	Prism.languages.tex = Prism.languages.latex;
	Prism.languages.context = Prism.languages.latex;
}(Prism));
